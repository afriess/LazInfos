%%####################################################################
%    Copyright @ 2007,2008 Andreas Frieß (Friess)
%    Permission is granted to copy, distribute and/or modify this document
%    under the terms of the GNU Free Documentation License, Version 1.2
%    or any later version published by the Free Software Foundation;
%    with no Invariant Sections, no Front-Cover Texts, and no Back-Cover Texts.
%    A copy of the license is included in the section entitled ``GNU
%    Free Documentation License''.
%%####################################################################
% Created: 02.01.2008
% @cvs($Date:  $)
% @cvs($Rev: $)
% @cvs($Author: af0815 $)
% @cvs($URL: $)
%%####################################################################
\subsection{Pascal}
\subsubsection{Mit Typen arbeiten}\footnote{Aus dem Thread \textsl{http://www.lazarusforum.de/viewtopic.php?f=55\&t=4354}}
Im folgenden Beispiel wird durch die einzelnen Werte gegangen. Bei der \textit{for-Schleife} werden dabei die richtigen Grenzen beachtet. Durch die Benutzung der Bibliothek \textit{Typeinfo} ist der Zugriff auf die gespeicherten TypenInformationen möglich.
\begin{verbatim}
Uses Typinfo;

Type TSpielfarben = (rot, gruen, blau, gelb);


procedure TForm1.Button1Click(Sender: TObject);
var x  : TSpielfarben;
      s : string;
begin
  memo1.clear;
  for x := low(x) to high(x) do
  begin
    s := GetEnumName(typeinfo(TSpielfarben),ord(x));
    Memo1.Append(s);
  end;
end;

\end{verbatim}



\verb|Version: $LastChangedRevision: 59 $ |\footnote{ Autor: Andreas Frieß\\Lizenz: GFDL}