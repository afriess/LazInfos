%%####################################################################
%    Copyright @ 2007,2008 Andreas Frie� (Friess)
%    Permission is granted to copy, distribute and/or modify this document
%    under the terms of the GNU Free Documentation License, Version 1.2
%    or any later version published by the Free Software Foundation;
%    with no Invariant Sections, no Front-Cover Texts, and no Back-Cover Texts.
%    A copy of the license is included in the section entitled ``GNU
%    Free Documentation License''.
%%####################################################################
% Created: 02.01.2008
% @cvs($Date:  $)
% @cvs($Rev: $)
% @cvs($Author: af0815 $)
% @cvs($URL: $)
%%####################################################################
\subsection{Pascal}
\subsubsection{Mit Typen arbeiten}\footnote{Aus dem Thread \textsl{http://www.lazarusforum.de/viewtopic.php?f=55\&t=4354}}
Im folgenden Beispiel wird durch die einzelnen Werte gegangen. Bei der \textit{for-Schleife} werden dabei die richtigen Grenzen beachtet. Durch die Benutzung der Bibliothek \textit{Typeinfo} ist der Zugriff auf die gespeicherten TypenInformationen m�glich.
\begin{verbatim}
Uses Typinfo;

Type TSpielfarben = (rot, gruen, blau, gelb);


procedure TForm1.Button1Click(Sender: TObject);
var x  : TSpielfarben;
      s : string;
begin
  memo1.clear;
  for x := low(x) to high(x) do
  begin
    s := GetEnumName(typeinfo(TSpielfarben),ord(x));
    Memo1.Append(s);
  end;
end;

\end{verbatim}


\subsection{Oberfl�che}
\subsubsection{Schrift auf einen Label �ndert sich nicht}
\footnote{Aus dem Thread \textsl{http://www.lazarusforum.de/viewtopic.php?p=16109\#16109}}
\begin{description}
	\item[Q:]�ndere ich den Wert der Eigenschaft \"font.size\" �ndert sich die Eigenschaft \"font.height\" im Gegenzug und scheint meine �nderung ausgleichen zu wollen. Die Schriftgr��e �ndert sich aber nicht. Autosize auf false setzen alleine hilft auch nicht.
	\item[A:]Dein Problem tritt auf, wenn du noch keine Schriftart ausgew�hlt hast.
Um das zu tun gehst du wie folgt vor:
Klicke auf das Label deiner Wahl. Im Objektinspektor erscheint eine Eigenschaft \"Font\" mit nem + davor. Wenn du auf die Eigenschaft Font (nicht aufs +) klickst, erscheint ein Feld, dahinter \"...\"
\\
Nun auf \"...\" klicken und ne Schriftart mit beliebiger Schriftgr��e festlegen.
Wenn die Schriftart nun mehr als nur eine Schriftgr��e erlaubt, kannst du nun auch �ber Font.Size die Gr��e der Labelschrift festlegen.   
\end{description}




\verb|Version: $LastChangedRevision: 59 $ |\footnote{ Autor: Andreas Frie�\\Lizenz: GFDL}