%%####################################################################
%    Copyright @ 2007 Andreas Frie� (Friess)
%    Permission is granted to copy, distribute and/or modify this document
%    under the terms of the GNU Free Documentation License, Version 1.2
%    or any later version published by the Free Software Foundation;
%    with no Invariant Sections, no Front-Cover Texts, and no Back-Cover Texts.
%    A copy of the license is included in the section entitled ``GNU
%    Free Documentation License''.
%%####################################################################
% Created: 08.11.2007
% @cvs($Date: $)
% @cvs($Rev:  $)
% @cvs($Author: af0815 $)
% @cvs($URL: $)
%%####################################################################
\section[Data Access - Clientdatenbank Komponenten]{DataAccess}
%%####################################################################
\subsection{Beschreibung}
\subsubsection{Einleitung}
Die Komponenten auf der Seite 'Client Access' sind f�r die lokale\label{ClientAccess} Verwaltung von Datenbanken, auch lokale Datenbanken genannt, zust�ndig. \parpic[sl][r]{\includegraphics[width=0.3\textwidth]{Kapitel/datenbanken/pics/DataAccess}}
\label{fig:LazDataAccess01} F�r die Erstellung, Verwaltung und Zugriff auf Dbase und Foxpro Datenbanken ist die Komponenete 'TDBF' zust�ndig. F�r Datenbanken direkt im Speicher ist die Komponenete 'MEM', f�r Datenbanken mit eine festen Format (CSV) die Komponente 'FIXED' und Textbasierende Datens�tze mittels der Komponente 'SDF'.
F�r Serverdatenbanken sind die Komponenten auf der Seite 'SQLdb'\footnote{Siehe Kapitel \ref{SQLdb} auf Seite \pageref{SQLdb}} zu verwenden.

Die Komponente 'TDatasource' dient unter andern als Verbindung zu den grafischen Komponenten.

\subsection{TDatasource}

\subsection{SDF}
%  public
%    constructor Create(AOwner: TComponent); override;
%  published
%    property Delimiter: Char read FDelimiter write SetDelimiter;
%    property FirstLineAsSchema: Boolean read FFirstLineAsSchema write SetFirstLineAsSchema;
%  end;
%
\subsection{FIXED}
%  public
%    constructor Create(AOwner: TComponent); override;
%    destructor  Destroy; override;
%    function  GetFieldData(Field: TField; Buffer: Pointer): Boolean; override;
%    procedure RemoveBlankRecords; dynamic;
%    procedure RemoveExtraColumns; dynamic;
%    procedure SaveFileAs(strFileName : String); dynamic;
%    property  CanModify;
%    procedure LoadFromStream(Stream :TStream);
%    procedure SavetoStream(Stream :TStream);
%  published
%    property FileMustExist: Boolean read FFileMustExist write SetFileMustExist;
%    property ReadOnly: Boolean read FReadOnly write SetReadOnly;
%    property FileName : TFileName read FFileName write SetFileName;
%    property Schema: TStringList read FSchema write SetSchema;
%    property TrimSpace: Boolean read FTrimSpace write SetTrimSpace default True;
%    property FieldDefs;
%    property Active;
%    property AutoCalcFields;
%    property Filtered;
%    property BeforeOpen;
%    property AfterOpen;
%    property BeforeClose;
%    property AfterClose;
%    property BeforeInsert;
%    property AfterInsert;
%    property BeforeEdit;
%    property AfterEdit;
%    property BeforePost;
%    property AfterPost;
%    property BeforeCancel;
%    property AfterCancel;
%    property BeforeDelete;
%    property AfterDelete;
%    property BeforeScroll;
%    property AfterScroll;
%//    property BeforeRefresh;
%//    property AfterRefresh;
%    property OnCalcFields;
%    property OnDeleteError;
%    property OnEditError;
%    property OnFilterRecord;
%    property OnNewRecord;
%    property OnPostError;

\subsection{TDBF}

\subsection{MEM}
%\subsubsection{Datentypen}
%\begin{verbatim}
%   ftString :   result:=FieldDefs.Items[FieldNo-1].Size+1;
%   ftBoolean :  result:=SizeOf(Wordbool);
%   ftFloat :    result:=SizeOf(Double);
%   ftLargeInt : result:=SizeOf(int64);
%   ftSmallInt : result:=SizeOf(SmallInt);
%   ftInteger :  result:=SizeOf(Integer);
%   ftDate :     result:=SizeOf(TDateTime);
%   ftTime :     result:=SizeOf(TDateTime);
%   ftDateTime : result:=SizeOf(TDateTime);
%\end{verbatim}

%
%  public
%    constructor Create(AOwner:tComponent); override;
%    destructor Destroy; override;
%    procedure CreateTable;
%
%    Function  DataSize : Integer;
%
%    procedure Clear(ClearDefs : Boolean);
%    procedure Clear;
%    Procedure SaveToFile(AFileName : String);
%    Procedure SaveToFile(AFileName : String; SaveData : Boolean);
%    Procedure SaveToStream(F : TStream);
%    Procedure SaveToStream(F : TStream; SaveData : Boolean);
%    Procedure LoadFromStream(F : TStream);
%    Procedure LoadFromFile(AFileName : String);
%    Procedure CopyFromDataset(DataSet : TDataSet);
%    Procedure CopyFromDataset(DataSet : TDataSet; CopyData : Boolean);
%
%    Property Modified : Boolean Read FModified;
%
%  published
%    Property FileName : String Read FFileName Write FFileName;
%    property Filtered;
%    Property Active;
%    Property FieldDefs;
%    property BeforeOpen;
%    property AfterOpen;
%    property BeforeClose;
%    property AfterClose;
%    property BeforeInsert;
%    property AfterInsert;
%    property BeforeEdit;
%    property AfterEdit;
%    property BeforePost;
%    property AfterPost;
%    property BeforeCancel;
%    property AfterCancel;
%    property BeforeDelete;
%    property AfterDelete;
%    property BeforeScroll;
%    property AfterScroll;
%    property OnDeleteError;
%    property OnEditError;
%    property OnNewRecord;
%    property OnPostError;
%    property OnFilterRecord;
%  end;
\subsection{SQLite}
Diese Komponente ist standardm�ssig nicht vorhanden. Dazu mu� erst das Paket 'sqlite3laz 0.3'oder besser installiert werden.  \parpic[sl][r]{\includegraphics[width=0.3\textwidth]{Kapitel/datenbanken/pics/DataAccessLite}}
\label{fig:LazDataAccess02}Nach dem neu erstellen der IDE, befindet sich die Komponente hier.


%    function BookmarkValid(ABookmark: TBookmark): Boolean; override;
%    function CompareBookmarks(Bookmark1, Bookmark2: TBookmark): Longint; override;
%    function GetFieldData(Field: TField; Buffer: Pointer): Boolean; override;
%    function GetFieldData(Field: TField; Buffer: Pointer; NativeFormat: Boolean): Boolean; override;
%    function Locate(const KeyFields: string; const KeyValues: Variant; Options: TLocateOptions) : Boolean; override;
%    function LocateNext(const KeyFields: string; const KeyValues: Variant; Options: TLocateOptions) : Boolean;
%    function Lookup(const KeyFields: string; const KeyValues: Variant; const ResultFields: string): Variant;override;
%    // Additional procedures
%    function ApplyUpdates: Boolean;
%    function CreateTable: Boolean;
%    function CreateTable(const ATableName: String): Boolean;
%    procedure ExecCallback(const ASql: String; UserData: Pointer = nil);
%    procedure ExecSQLList;
%    procedure ExecuteDirect(const ASql: String);virtual;abstract;
%    procedure QueryUpdates(RecordStates: TRecordStateSet; Callback: TQueryUpdatesCallback; UserData: Pointer=nil);
%    function QuickQuery(const ASql:String):String;overload;
%    function QuickQuery(const ASql:String;const AStrList: TStrings):String;overload;
%    function QuickQuery(const ASql:String;const AStrList: TStrings;FillObjects:Boolean):String;virtual;abstract;overload;
%    procedure RefetchData;
%    function TableExists: Boolean;
%    function TableExists(const ATableName:String):Boolean;
%    function UpdatesPending: Boolean;
%    property ExpectedAppends: Integer read FExpectedAppends write SetExpectedAppends;
%    property ExpectedUpdates: Integer read FExpectedUpdates write SetExpectedUpdates;
%    property ExpectedDeletes: Integer read FExpectedDeletes write SetExpectedDeletes;
%    property IndexFields[Value: Integer]: TField read GetIndexFields;
%    property RowsAffected: Integer read GetRowsAffected;
%    property ReturnCode: Integer read FReturnCode;
%    property SqliteHandle: Pointer read FSqliteHandle;
%    property SqliteVersion: String read GetSqliteVersion;
%    property SQLList:TStrings read FSqlList;
%   published
%    property AutoIncrementKey: Boolean read FAutoIncrementKey write FAutoIncrementKey;
%    property IndexFieldNames: string read FIndexFieldNames write FIndexFieldNames;
%    property FileName: String read FFileName write SetFileName;
%    property OnCallback: TSqliteCallback read FOnCallback write FOnCallback;
%    property PrimaryKey: String read FPrimaryKey write FPrimaryKey;
%    
%    property TableName: String read FTableName write FTableName;   
%    property MasterSource: TDataSource read GetMasterSource write SetMasterSource;
%    property MasterFields: string read GetMasterFields write SetMasterFields;
%    
\subsubsection{ExecSQL}
\begin{description}
  \item \texttt{procedure ExecSQL(const ASql:String);}\\F�hrt den in ASql definierten String aus, liefert aber keine Datenmenge zur�ck.
  \item \texttt{procedure ExecSQL;}\\F�hrt die in der Eigenschaft SQL definierten SQL-Befehle aus, liefert aber keine Datenmenge zur�ck.
  \begin{description}
    \item Methode von TSqlite3Dataset
  \end{description}
\end{description}

\subsubsection{ApplyUpdates}
\begin{description}
  \item \texttt{function ApplyUpdates: Boolean;}\\�bertr�gt die �nderungen von der Datenmenge in die Datenbank.
  \begin{description}
    \item Methode von TSqlite3Dataset>TCustomSqliteDataset
  \end{description}
\end{description}

\subsubsection{ExecSQL}
\begin{description}
  \item \texttt{procedure ExecSQL(const ASql:String);}\\F�hrt den in ASql definierten String aus, liefert aber keine Datenmenge zur�ck.
  \item \texttt{procedure ExecSQL;}\\F�hrt die in der Eigenschaft SQL definierten SQL-Befehle aus, liefert aber keine Datenmenge zur�ck.
  \begin{description}
    \item Methode von TSqlite3Dataset
  \end{description}
\end{description}

\subsubsection{Active}
\begin{description}
  \item \texttt{property Active: Boolean read GetActive write SetActive default False;}\\ Sagt aus ob die Komponente derzeit aktiv ist und eine Datenmenge zur�ckliefert. Ein setzen auf 'true' entspricht der Methode 'Open' und ein setzen auf 'false' entspricht der Methode 'Close'.
  \begin{description}
    \item Eigenschaft von TSqlite3Dataset>TCustomSqliteDataset>TDataSet
  \end{description}
  \begin{description}
    \item Zugriff: Lesend und schreibend
    \item Defaultwert: false
  \end{description}
\end{description}

\subsubsection{SaveOnClose}
\begin{description}
  \item \texttt{property SaveOnClose: Boolean read FSaveOnClose write FSaveOnClose;}\\Wenn true, dann wird beim Schliessen der Datenmenge automatisch ein ApplyUpdates durchgef�hrt.
  \begin{description}
    \item Eigenschaft von TSqlite3Dataset>TCustomSqliteDataset
  \end{description}
  \begin{description}
    \item Zugriff: Lesend und schreibend
  \end{description}
\end{description}

\subsubsection{SaveOnRefetch}
\begin{description}
  \item \texttt{property SaveOnRefetch: Boolean read FSaveOnRefetch write FSaveOnRefetch;}\\Wenn true, dann wird vor dem Holen neuer Datenmengen automatisch ein ApplyUpdates durchgef�hrt.
  \begin{description}
    \item Eigenschaft von TSqlite3Dataset>TCustomSqliteDataset
  \end{description}
  \begin{description}
    \item Zugriff: Lesend und schreibend
  \end{description}
\end{description}

\subsubsection{SQL}
\begin{description}
  \item \texttt{property SQL: String read FSql write FSql;}\\In SQL befindet sich das SQL Statement was ausgef�hrt werden soll.
  \begin{description}
    \item Eigenschaft von TSqlite3Dataset>TCustomSqliteDataset
  \end{description}
  \begin{description}
    \item Zugriff: Lesend und schreibend
  \end{description}
\end{description}

\verb|Version: $$ |\footnote{ Autor: Andreas Frie�\\Lizenz: GFDL}