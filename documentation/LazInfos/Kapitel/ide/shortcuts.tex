%%####################################################################
%    Copyright @ 2007 Andreas Frie� (Friess)
%    Permission is granted to copy, distribute and/or modify this document
%    under the terms of the GNU Free Documentation License, Version 1.2
%    or any later version published by the Free Software Foundation;
%    with no Invariant Sections, no Front-Cover Texts, and no Back-Cover Texts.
%    A copy of the license is included in the section entitled ``GNU
%    Free Documentation License''.
%%####################################################################
% Created: 21.08.2007
% @cvs($Date: 2007-11-18 17:51:21 +0000 (Sun, 18 Nov 2007) $)
% @cvs($Rev: 65 $)
% @cvs($Author: af0815 $)
% @cvs($URL: file:///svn/p/lazsnippets/code/trunk/dokumentation/LazSnippets/Kapitel/ide/shortcuts.tex $)
%%####################################################################
\subsection[Tastenkombinationen]{Tastenkombinationen}
Die aktuelle Quelle f�r die Tastenkombinationen \footnote{/Lazarus\_\-IDE\_Tools/\-de\#.C3.9Cbersichts\-tabelle\_der\_IDE\_Tasten\-kombinationen} ist die Lazarus-Homepage\cite{wi.Tast}.

\begin{table}[htbp]
%	\centering
		\begin{tabular}[ht]{|l|l|}
      \hline
      Datei & \\
      \hline
      Tastenk�rzel & Erkl�rung \\
      \hline
      Strg+O & Datei �ffnen\\
      Strg+S & Datei speichern\\
      Umsch+Strg+S & alle Dateien speichern\\
      Strg+P & Drucken\\
      \hline
		\end{tabular}
  \caption{IDE - Tastenkombinationen Datei}
  \label{tab:IDETastenkombDatei} 
\end{table}
\begin{table}[htbp]
%	\centering
		\begin{tabular}[ht]{|l|l|}
      \hline
      Bearbeiten & \\
      \hline
      Tastenk�rzel & Erkl�rung \\
      \hline
      Strg+Z & R�ckg�ngig\\
      Umsch+Strg+Z & Wiederholen\\
      Strg+X & Auswahl ausschneiden\\
      Strg+C & Auswahl kopieren\\
      Strg+V & Auswahl einf�gen\\
      Strg+I & Auswahl einr�cken\\
      Strg+U & Auswahl ausr�cken\\
      Strg+A & Alles ausw�hlen\\
      Umsch+Strg+D & \$IFDEF einf�gen\\
      Umsch+Strg+C & Kodevervollst�ndigung (Class Completion) \\
      \hline
		\end{tabular}
  \caption{IDE - Tastenkombinationen Bearbeiten}
  \label{tab:IDETastenkombBearbeiten} 
\end{table}
\begin{table}[htbp]
%	\centering
		\begin{tabular}[ht]{|l|l|}
      \hline
      Suchen & \\
      \hline
      Tastenk�rzel & Erkl�rung \\
      \hline
      F3 & n�chstes suchen\\
      Umsch+F3 & vorheriges suchen\\
      Strg+R & Ersetzen\\
      Strg+E & Inkremtielle Suche\\
      Strg+G & Zu Zeile springen\\
      Strg+H & Zur�ckspringen\\
      Umsch+Strg+H & Vorw�rtsspringen\\
      Strg+F8 & Zum n�chsten Fehler springen\\
      Umsch+Strg+F8 & Zum vorherigen Fehler springen\\
      Alt+Up & Deklaration unter Cursor suchen\\
      Strg+Enter & Datei unter Cursor �ffnen\\
      Umsch+Strg+G & Prozedur Liste\\
      \hline
		\end{tabular}
  \caption{IDE - Tastenkombinationen Suchen}
  \label{tab:IDETastenkombSuchen} 
\end{table}
\begin{table}[htbp]
%	\centering
		\begin{tabular}[ht]{|l|l|}
      \hline
      Ansicht & \\
      \hline
      Tastenk�rzel & Erkl�rung \\
      \hline
      F11 & Objektinspektor anzeigen\\
      Strg+F12 & Unitliste anzeigen\\
      Umsch+F12 & Formularliste anzeigen\\
      F12 & Formulare/Unit Anzeige umschalten\\
      Strg+Alt+F & Suchergebnisse anzeigen\\
      Strg+Alt+B & �berwachte Ausdr�cke\\
      Strg+Alt+W & Haltepunkte\\
      Strg+Alt+L & Lokale Variablen\\
      Strg+Alt+S & Aufruf Stack\\
      \hline
		\end{tabular}
  \caption{IDE - TastenkombinationenAnsicht}
  \label{tab:IDETastenkombAnsicht} 
\end{table}
\begin{table}[htbp]
%	\centering
		\begin{tabular}[ht]{|l|l|}
      \hline
      Projekt & \\
      \hline
      Tastenk�rzel & Erkl�rung \\
      \hline
      Strg+F11 & Projekt �ffnen\\
      Umsch+Strg+F11 & Projekt Einstelllungen\\
      Umsch+F11 & Datei ins Projekt �bernehmen\\
      \hline
		\end{tabular}
  \caption{IDE - Tastenkombinationen Projekt}
  \label{tab:IDETastenkombDivProjekt} 
\end{table}
\begin{table}[htbp]
%	\centering
		\begin{tabular}[ht]{|l|l|}
      \hline
      Start & \\
      \hline
      Tastenk�rzel & Erkl�rung \\
      \hline
      Strg+F9 & Erstellen\\
      F9 & Start\\
      F7 & Ein Schritt\\
      F4 & Start bis Cursor\\
      Strg+F2 & Halt \\
      Strg+F7 & Pr�fen/�ndern\\
      Strg+F5 & �berwachung hinzuf�gen\\
      \hline
		\end{tabular}
  \caption{IDE - Tastenkombinationen Start}
  \label{tab:IDETastenkombStart} 
\end{table}
\begin{table}[htbp]
%	\centering
		\begin{tabular}[ht]{|l|l|}
      \hline
      Divers & \\
      \hline
      Tastenk�rzel & Erkl�rung \\
      \hline
      Strg+Click & springt zur Deklaration eines Typen oder Variablen \\
      Strg+Shift+Upe & schaltet zwischen Definition und Rumpf um \\
      Strg+J & Code-Schablonen \\
      Strg+Space & Bezeichnervervollst�ndigung \\
      Strg+W & Word Completion \\
      Strg+Shift+Space & Parameter Hinweise \\
      \hline
		\end{tabular}
  \caption{IDE - Tastenkombinationen}
  \label{tab:IDETastenkombDiv} 
\end{table}
\verb|Version: $LastChangedRevision: 65 $ |\footnote{ Autor: Andreas Frie�\\Lizenz: GFDL}