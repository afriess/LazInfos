%%####################################################################
%    Copyright @ 2007 Andreas Frie� (Friess)
%    Permission is granted to copy, distribute and/or modify this document
%    under the terms of the GNU Free Documentation License, Version 1.2
%    or any later version published by the Free Software Foundation;
%    with no Invariant Sections, no Front-Cover Texts, and no Back-Cover Texts.
%    A copy of the license is included in the section entitled ``GNU
%    Free Documentation License''.
%%####################################################################
% Created: 18.08.2007
% @cvs($Date: 2009-07-23 13:43:23 +0000 (Thu, 23 Jul 2009) $)
% @cvs($Rev: 95 $)
% @cvs($Author: monta-lf $)
% @cvs($URL: file:///svn/p/lazsnippets/code/trunk/dokumentation/LazSnippets/Kapitel/allgemein/Versionskontrolle.tex $)
%%####################################################################

\subsection[Versionskontrolle]{Versionskontrolle}
Warum ein Versionsmanagment �berhaupt verwenden? Es ist doch viel einfacher den Sourcecode und die ben�tigten Dateien einfach auf der Festplatte zu haben und von Zeit zu Zeit macht man davon ein Archiv. Diesem Archiv gebe ich ganz einfach einen guten Titel und die Sache hat sich.

Das kann bei einem einzelnen Entwickler durchaus so noch funktionieren. Wenn man aber in einem Team arbeiten mu�, so wird es schon hier zu Problemen kommen. Denn jeder Entwickler hat seine eigene Version und mu� die �nderungen in das Projekt ein pflegen. Was ist aber wenn gerade 2 Entwickler an derselben Datei arbeiten. Der eine bringt die �nderungen ein, der andere bekommt die �nderung aber auch nicht mit und bringt seinerseits seine �nderungen ein. Somit sind die �nderungen des ersten Entwicklers unbemerkt verloren und das Projekt somit nicht konsistent. Weiters ist die Verfolgung der �nderungen fast, bzw. nur schwer m�glich. Um dieses Szenario zu vermeiden und die arbeit im Team zu vereinfachen haben sich im laufe der Jahre Versionsmanagmentsysteme entwickelt. Da wir hier nur SVN einsetzen werden, will ich mich hier auf die grundlegende Bedienung von SVN beschr�nken.

\subsubsection[svn Kommandozeile]{svn Kommandozeile}
SVN von der Kommandozeile aus zu bedienen ist eine der universellsten M�glichkeiten, das diese Art damit zu arbeiten auf allen Plattformen (Unix, Linux, Windows, MacOS, ...) vorhanden ist. Vor allen wenn man nur Projekte aus dem SVN holt und auffrischt ist die Syntax relativ einfach. Die daf�r notwendigen Befehle sind hier kurz vorgestellt.

\paragraph[SVN holen und Informationen anzeigen]{SVN holen und Informationen anzeigen}
\begin{description}
	\item[svn checkout URL [Pfad] ] Mit diesem Kommando wird von der URL der gespeicherte Inhalt aus dem Repository geholt und an der Stelle gespeichert die durch den Pfad angegeben ist. Fehlt die Angabe des Pfades, so wird das aktuelle Verzeichnis genommen.
\end{description}	
\begin{description}
	\item[svn co URL [Pfad] ] ist die Kurzform von "`checkout"', Erkl�rung siehe dort.
\end{description}	
\begin{description}
	\item[svn help ] Gibt die Onlinehilfe zu SVN aus.
\end{description}	
\begin{description}
	\item[svn list URL] und
	\item[svn list Pfad ] Zeigt zus�tzliche Informationen zur URL beziehungsweise zum Pfad an.
\end{description}	
\begin{description}
	\item[svn log URL] und
	\item[svn log Pfad ] Zeigt die Loginformationen zur URL beziehungsweise zum Pfad an.
\end{description}
\begin{description}
	\item[svn update] Aktualisiert die lokale Kopie und l�d alle �nderungen, welche in der Zwischenzeit von anderen entwicklern am Repository vorgenohmen wurden.
\end{description}	
\begin{description}
	\item[svn revert Pfad] Entfernt die �nderungen sei der letzten Revision und stellt einen unver�nderten Zustand wieder her. 
\end{description}	

\paragraph[�nderungen �bertragen]{�nderungen �bertragen}
\begin{description}
	\item[svn add Datei] F�gt eine Datei in der Arbeitskopie zur Versionskontrolle hinzu. Die Datei wird jedoch nicht direkt in das Repository geladen, sondern lediglich markiert.
\end{description}
\begin{description}
	\item[svn commit \[-m"Beschreibung..."\]] Legt eine neue Revision an und �bertr�gt alle �nderungen an Dateien, welche sich bereits unter Versionkontrolle befinden bzw. vorher mit add markiert wurden. Der Parameter -m erlaubt es, direkt eine Beschreibung der �nderungen anzugeben, welche sp�ter ein Nachverfolgen der �nderungen erm�glicht.
\end{description}

\verb|Version: $LastChangedRevision: 95 $ |\footnote{ Autor: Andreas Frie�\\Lizenz: GFDL}